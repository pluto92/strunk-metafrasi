\translationsetup
    {Joseph Addison}
    {1711}
    {The Spectator}
    {Μανώλης Κυζάλας}

\pagestyle{texts}
\chapter[%
    \originalauthor\ -- \emph{\translatedtitle}\ \yearpublished\\%
    {\normalfont\emph{μετάφραση:} \maintranslator}%
]{\originalauthor}

\begin{preface}
    Ενάς από τους σπουδαιότερους συγγραφείς και ανθρώπους των γραμμάτων, ο Addison (1672-1719) συνείσφερε κείμενά του σε πολλά από τα περιοδικά της εποχής, όπως το \emph{The Tatler, The Spectator} και το \emph{The Guardian}, που δημοσιεύονταν από τον φίλο του Richard Steele, από το 1709 εώς το 1713. Ιδιαίτερα σημαντικές είναι οι συνεισφορές του Addison στο Spectator που κρατούσε μετριοπαθή στάση στους πολιτικούς διαξιφισμούς της εποχής. Στα κείμενα του ο Addison περιγράφει γλαφυρά τον τρόπο ζωής και συμπεριφοράς των σύγχρονών του. Τα έξυπνα κείμενά του είχαν σημαντική επιρροή στην χώρο της κριτικής, όχι μόνο σε Άγγλους συγγραφείς, αλλά και σε Γάλλους και Γερμανούς.
\end{preface}

\begin{center}
    \textbf{\translatedtitle}

    [αποσπάσματα]

    μετάφραση: \maintranslator
\end{center}

\begin{center}
    \emph{Τρίτη, 6 Μαρτίου, 1711.}

    \emph{Spectatum admissi risum teneatis?}%
    \footnote{Θα μπορούσατε, φίλοι μου, εάν είχατε την ευκαιρία για μία ιδιωτική θέαση, να μην γελάσετε; --- Οράτιος, \emph{Ars Poetica}}
\end{center}

\noindent
Θα επιτρέπαμε σε μία όπερα να είναι υπερβολικά πληθωρική με τα στολίδια της, αφού ο μόνος στόχος της είναι να ευχαριστεί τις αισθήσεις, και να διατηρήσει την προσοχή ενός νωχελικού κοινού. Όπως πρόταση, όμως, η κοινή λογική, δεν θα πρέπει να υπάρχει τίποτα στη σκηνή και στις μηχανές που χρησιμοποιούνται που να φαίνεται παιδικό η άτοπο. Πόσο θα γελούσαν οι πνευματικοί άνθρωποι του καιρού του βασιλιά Καρόλου, εάν έβλεπαν τον Nicolini εκτεθειμένο σε μία καταιγίδα, ενώ ήταν ντυμένος με στολή ερμίνας, ενώ έπλεε σε μία θάλασσα από χαρτόνι! Πόσο θα τους χλεύαζαν εάν έβλεπαν ζωγραφισμένος δράκους να πετάνε φωτιές, μαγεμένα άρματα ζωγραφισμένα από φλαμανδικές φοράδες, και πραγματικούς καταρράκτες να χύνονται πάνω σε τεχνητά τοπία. Με ελάχιστη ικανότητα στην κριτική θα μπορούσαμε να καταλάβουμε πως οι σκιές και η πραγματικότητα δεν πρέπει να αναμειγνύονται μέσα στο ίδιο έργο· πως οι σκηνές που είναι σχεδιασμένες να παρουσιάζουν την φύση θα πρέπει να είναι γεμάτες αναπαραστάσεις και όχι με τα πραγματικά αντικείμενα αυτούσια. Εάν κάποιος ήθελε να αναπαραστήσει μία πλατιά πεδιάδα με αγέλες και κοπάδια, θα ήταν γελοίο να ζωγραφίσει την πεδιάδα πάνω στη σκηνή αλλά να πλημμυρίσει το σανίδι με πρόβατα και βόδια. Αν έκανε αυτό θα συνδύαζε ασύνδετα πράγματα και θα και θα έφτιαχνε ένα στήσιμο σκηνής εν μέρει αληθινό και εν μέρει φανταστικό. Θα πρότεινα αυτά που είπα παραπάνω σε όλους τους σκηνοθέτες, όπως και σε όλους τους θαυμαστές της σύγχρονης όπερας.

Όπως περπατούσα στους δρόμους πριν από ένα δεκαπενθήμερο, είδα έναν συνηθισμένο τύπο να κουβαλάει ένα κλουβί γεμάτο με μικρά πουλιά στους ώμους του· και καθώς αναρωτιόμουνα τι τα χρειάζεται, εντελώς τυχαία συναντήθηκε με ένα γνωστό του που είχε την ίδια απορία. Αφού τον ρώτησα τι κουβαλάει στους ώμους του, εκείνος του απάντησε πως αγοράζει σπουργίτια για την όπερα. Σπουργίτια για την όπερα! Είπε ο φίλος του, ενώ του τρέχανε τα σάλια, για να τα ψήσουνε; Όχι, όχι, λέει ο άλλος, Είναι για να Μπουν στο τέλος της πρώτης πράξεις και να πετάξουν εκτός σκηνής.

Αυτός ο παράξενος διάλογος μου κίνησε τόσο πολύ την περιέργεια που αγόρασα εισιτήρια για την όπερα, που πίστευα πως τα σπουργίτια θα τραγουδήσουν σε κάποιο ευχάριστο δάσος. Παρατηρώντας καλύτερα, βέβαια, κατάλαβα πως τα σπουργίτια προσπάθησα να με ξεγελάσουν όπως ξεγέλασε ο Sir Martin Mar-all την ερωμένη του, αφού ενώ αυτά πετούσαν μπροστά στα μάτια μου, η μουσική ερχόταν πίσω από τις κουρτίνες, με φλάουτα και καλέσματα πουλιών. Την ίδια στιγμή που έκανε αυτή την ανακάλυψη, έμαθα επίσης, από τις συζητήσεις των ηθοποιών, πως υπήρχαν μεγάλα σχέδια για την βελτίωση αυτής της όπερας. Είχε προταθεί να ρίξουν ένα μέρος του τοίχου και να ξαφνιάσουν το κοινό με εκατό άλογα και πως υπήρχε ένα εγχείρημα στα σκαριά, να φέρουν το New River μέσα στο κτίριο, ώστε να φτιάξουν πίδακες και άλλες κατασκευές με το νερό. Αυτό το εγχείρημα όπως έχω ακούσει, θα αναβληθεί μέχρι και το καλοκαίρι, όπου η ψύχρα και η υγρασία από τους πίδακες και τους καταρράχτες θα είναι πιο ευπρόσδεκτη από ανθρώπους επιπέδου. Εν το μεταξύ, για ένα πιο ταιριαστό θέαμα στον χειμώνα, η όπερα του Rinaldo είναι γεμάτη με βροντές, αστραπές και φωτιές. Το κοινό μπορεί να τα παρακολουθήσει χωρίς να κρυώσει, και σίγουρα δεν θα διατρέχει πολύ μεγάλο κίνδυνο να καεί· αφού υπάρχουν πολλές μηχανές γεμάτες νερό, και έτοιμες για λειτουργία σε περίπτωση τέτοιων ατυχημάτων. Παραταύτα, επειδή ο ιδιοκτήτης του θεάτρου είναι πολύ καλός μου φίλος, ελπίζω να ήταν αρκετά σοφός ώστε να το ασφαλίσει πριν παιχτεί αυτή η όπερα.

Σίγουρα αυτές οι σκηνές θα έχουν πάρα πολύ ενδιαφέρον, αφού τις έχουν γράψει δύο ποιητές διαφορετικής εθνικότητες και τις απαρτίζουν δύο μάγοι διαφορετικού φύλλου. Η Armida (όπως μαθαίνουμε από τον διάλογο) είναι αμαζόνα μάγισσα, ενώ ο φτωχός Signor Cassani (όπως μαθαίνουμε από τον χαρακτήρα του) είναι ένας χριστιανός μάγος (Mago Christiano). Πρέπει να ομολογήσω πως είμαι πολύ προβληματισμένος· πώς μία αμαζόνα μαθαίνει την τέχνη της μαύρης μαγίας, αλλά και πώς ένας καλός χριστιανός συνδιαλέγεται με τον διάβολο.

Ας αφουγκραστούμε και τους ποιητές μετά τους μάγους. Θα δώσω μία γεύση των Ιταλικών στις πρώτες γραμμές τις εισαγωγής. \en{Eccoti, benigno lettore, un parto di poche sere, che se ben nato di notte, non è pèro aborto di tenebre, mà si farà conoscere figliolo d'Apollo con qualche raggio di Parnasso}. \enquote{Δες, ευγενικέ αναγνώστη, τη γέννηση μερικών βραδιών, που, αν και είναι τα παιδιά της νύχτας, δεν είναι --- του σκοταδιού, αλλα θα φανερωθεί φανερωθεί ως ο γιος του Απόλονα με το φως του Παρνασού.} Στη συνέχεια αποκαλεί τον Mynheer Hendel \enquote{Ορφέα της εποχής μας}, και στη συνέχεια μας γνωστοποιεί, με την ίδια αιθεριότητα στο ύφος του, πως έγραψε την όπερα, μόλις σε ένα δεκαπενθήμερο. Αυτό είναι το πνέυμα αυτών, στων οποίων το γούστο τόσο φιλόδοξα επαναπαυόμαστε. Η αλήθεια Του ζητήματος είναι πως οι μεγαλύτεροι Ιταλοί συγγραφείς της εποχής μας εκφράζονται με ένα τόσο διανθισμένο λεξιλόγιο, και με μία τόσο ανιαρή περιφραστικότητα, που στην χώρα μας δεν συνηθίζει κανείς παρά μόνο οι πιο σχολαστικοί ακαδημαϊκοί, ενώ ταυτόχρονα γεμίζουν τα γραπτά τους με τόσο φτωχή φαντασία και ματαιοδοξία, που η νεολαία μας θα ντρεπόταν ακόμα και πριν τα πρώτα δύο χρόνια στο πανεπιστήμιο. Πολλοί θα βιαστούν να πουν πως είναι η διαφορά στη ευφυΐα που ευθύνεται για την διαφορά στο επίπεδο των έργων των δύο εθνών, μα για να αποδείξω πως δεν ισχύει κάτι τέτοιο, δεν χρειάζεται παρά να κοιτάξουμε στα γραπτά των παλιών Ιταλών, όπως του Cicero και του Βιργίλιου· Αυτό που θα βρούμε είναι πως οι άγγλοι συγγραφείς, στον τρόπο σκέψης τους και έκφρασής τους, είναι πολύ πιο κοντά στου παλιούς Ιταλούς, απ'ότι οι σύγχρονοι Ιταλοί που απλώς υποκρίνονται πως είναι. Και για τον ποιητή τον ίδιο, από του οποίου τα όνειρα δημιουργήθηκε αυτή η όπερα, θα συμφωνήσω απόλυτα με τον Monsieur Boileau, πως μία στροφή του Βιργίλιου, αξίζει όσο όλες οι χάντρες κι οι πούλιες του Tasso. 

Ας επιστρέψουμε όμως στα σπουργίτια. Υπήρξαν τόσες πτήσεις με τα αφηνιασμένα σπουργίτια σε αυτήν την όπερα, που είναι δύσκολο να βγάλουν πλέον όλα έξω. Μπαίνουν στα έργα που δεν έχουν καμία θέση, σε πολύ ακατάλληλες σκηνές· βρίσκονται σε κρεβατοκάμαρες κυριών, ή κάθονται στον θρόνο του βασιλία· και πόσο υποφέρουν τα κεφάλια του κοινού. Έμπιστες πηγές μου έχουν μεταφέρει πως μία φορά έγινε η προσπάθεια να ανέβει η ιστορία του Whittington και της γάτας του, και για την προσπάθεια αυτή, μάζεψαν πάρα πολλά ποντίκια. Ο κύριος Rich όμως, ο ιδιοκτήτης του θεάτρου, πολύ σοφά σκέφτηκε πως θα ήταν αδύνατο για τη γάτα να τα σκοτώσει όλα, και συνεπώς οι πρίγκιπες στην σκηνή θα είχαν τόσα ποντίκια όσο ο πρίγκιπας στο νησί, πριν την άφιξη της γάτας, και για αυτό δεν το επέτρεψε τελικά. Και φυσικά δεν μπορώ να τον κατηγορήσω, αφού όπως είπε, δεν ήξερε κάποιον από τους μουσικούς στην όπερα, τόσο ικανό όσο τον γνωστό pied piper, που έκανε όλα τα ποντικιά μίας μεγάλης πόλης στην γερμανία, να ακολουθήσουν τη μουσική του, και έτσι καθάρισε όλη την περιοχή από περίεργα μικρά ζώα.

Πριν τελειώσω αυτό το κείμενο, πρέπει να ενημερώσω τους αναγνώστες μου, ότι ακούγονται φήμες πως ο London και ο Wise (οι επιμελητές του κήπου του θεάτρου) θα στολίσουν την όπερα του Rinaldo και της Armida με έναν πορτοκαλόκηπο, και πως την επόμενη φορά που θα ανέβει η όπερα, τον ρόλο των πουλιών θα τον παίξουν αιγίθαλοι. Για αυτήν την εργολαβία δεν θα τσιγκουνευτούν ούτε στον κόπο τους ούτε στα λεφτά· όλα για χάρη της ευχαρίστης του κοινού.

\begin{center}
    \emph{Πέμπτη, 15 Μαρτίου, 1711.}

    \emph{Die mihi, nfills tu leo, qualis eris?}\footnote{Πες μου, αν ήσουν λιοντάρι, τι λιοντάρι θα ήσουν; --- Επιγράμματα [ΧΙΙ, xeii], Martialis}
\end{center}

Δεν υπάρχει τίποτα τα τελευταία χρόνια που να προσέφερε μεγαλύτερη ευχαρίστηση στην πόλη από την μάχη του Signor Nicolini με ένα λιοντάρι στην αγορά, που την έχουν παρακολουθήσει πολλές φορές με μεγάλη ικανοποίηση οι άρχοντες και οι ευγενείς του βασιλείου της Μεγάλης Βρετανίας. Όταν το ακούσαμε για πρώτη φορά, πιστέψαμε πως κάθε βράδυ θα στέλνανε ένα διαφορετικό εξημερωμένο λιοντάρι ώστε να το σκοτώσει ο Υδάσπης. Αυτή η αβασίμη φήμη επικράτησε στα υψηλά μέρη του θεάτρου, όπου πολλοί εκλεπτισμένοι πολιτικοί που βρίσκονταν στο κοινό, ψιθύριζαν πως το λιοντάρι ήταν πρώτος ξάδερφος του τίγρη που εμφανιζόταν στις μέρες του Βασιλιά William, και πως το θέατρο θα είχε στη διάθεσή του λιοντάρια για όλη τη σεζόν με λεφτά από δημόσιες δαπάνες. Πολλοί, μάλιστα, έκαναν εικασίες για μεταχείριση που που θα έχαιρε το λιοντάρι στα χέρια του Signor Nicolini· κάποιοι έλεγαν πως θα το σαγήνευε με το recitativo του, όπως ο Ορφέας τα άγρια κτήνη στον καιρό του, και μετά θα το χτυπούσε στο κεφάλι. Κάποιοι πίστευαν πως το λιοντάρι δεν θα τολμούσε να απλώσει χέρι πάνω σε έναν παρθένο, λόγω των συνεπειών για την εικόνα του στην κοινή γνώμη. Άλλοι, που υποτίθεται πως είχαν δει αυτήν την όπερα στην Ιταλία, είπαν στους φίλους τους, πως το λιοντάρι έλεγε κάποια λόγια στη βασιλική διάλεκτο των Ολλανδικών, και θα βρυχώταν δύο με τρεις φορές, συνοδευόμενο από basso continuo, πριν πέσει στα πόδια του Υδάσπη. Για να ξεκαθαρίσουν τα πράγματα, έβαλα στόχο να αναγνωρίσω αν αυτό το λιοντάρι ήταν το κτήνος που φαινόταν, ή αν ήταν όλα ψέματα. 

Πριν σας γνωστοποιήσω τις ανακαλύψεις πάνω στο ζήτημα, θα πρέπει να σας ενημερώσω, πως πέρυσι τον χειμώνα, καθώς περπατούσα τυχαία προς τα παρασκήνια, βρέθηκα μπροστά σε ένα τερατώδες ζώο που με τρόμαξε πολύ. Με μία περαιτέρω διερεύνηση, έγινε φανερό πως πρόκειται για ένα αφινιασμένο λιοντάρι. Αυτό, βλέποντας την έκπληξή στο πρόσωπό μου, μου είπε με γλυκιά φωνή, πως μπορώ να το πλησιάσω αν θέλω, \enquote{δεν θέλω να βλάψω κανέναν}, μου είπε. Το ευχαρίστησα πολύ και το προσπέρασα. Μετά από λίγο το είδα να ανεβαίνει στην σκηνή, να κάνει το μέρος του και να χειροκροτάτε θέρμα. Πολλοί παρατήρησαν πως είχε αλλάξει τον τρόπο που επιτελούσε τον ρόλο του, δύο με τρεις φορές από την πρώτη εμφάνισή του. Κάτι που φαίνεται πολύ λογικό, μόλις καταλάβει κανείς πως το λιοντάρι έχει αλλαχτεί τρεις φορές. Το πρώτο ίσα που μπορούσε να σβήσει ένα κερί με την αναπνοή του, και γιαυτό ήταν λίγο ευερέθιστο και το παράκανε με τον ρόλο του, και δεν πέθαινε πάνω στην όπως έπρεπε. Επίσης, παρατηρήθηκε πως κάθε φορά που έβγαινε στην σκηνή, ήταν πως το ανέβαινε η αυτοπεποίθηση κάθε φορά που έβγαζε τη στολή, και του ξέφευγαν μερικές κανονικές κουβέντες, λες και δεν έβαλε τα δυνατά του στην μάχη, και πως πόνεσε πέφτοντας στο σανίδι με την πλάτη τόσες φορές, ενώ θα πάλευε με τον κύριο Nicolini για οποιοδήποτε λόγο χωρίς τη στολή του λιονταριού πάνω του. Λόγω των παραπάνω, φάνηκε αναγκαίο να τον απολύσουν, και πιστεύεται πως εάν ξανά ανέβαινε στην σκηνή κάποια ζημία θα έκανε. Επίσης, λένε πως η στάση του δεν ήταν αρκετά σκυφτή, ούτε αρκετά όρθια, έτσι έμοιαζε περισσότερο σαν γέρος, παρά σαν λιοντάρι.

Το δεύτερο λιοντάρι ήταν ράφτης στο επάγγελμα, που εργαζόταν στο θέατρο και ήταν μετριοπαθής και φιλήσυχος χαρακτήρα. Εάν ο προηγούμενος ήταν πολύ έξαλλος, αυτός ήταν περισσότερο σαν πρόβατο· τόσο πολύ που μετά από ένα μικρό χρονικό διάστημα πάνω στην σκηνή, έπεφτε με το παραμικρό άγγιγμα από το Υδάσπης, χωρίς να τον παλέψει και να του δώσει την ευκαιρία για την ανάδειξη Ιταλικών τρικλοποδιών. Λέγεται, πως μία φορά έσκισε το ένδυμά του, αλλά το μόνο που κατάφερε είναι να δημιουργήσει περισσότερη εργασία για τον εαυτό του. Και δεν πρέπει να παραλείψω να πω, πως αυτός ήταν το ευγενικό λιοντάρι που συνάντησα.

Το παρόν λιοντάρι, όπως πληροφορούμε, είναι ένας ευγενής της επαρχίας, που το κάνει για να περάσει την ώρα του, και προτιμάει να μην μαθευτεί το όνομά του. Λέει, πολύ γενναιόδωρα, για να δικαιολογηθεί, πως δεν το κάνει για το κέρδος· απολαμβάνει μία αθώα ευχαρίστη από αυτό και είναι καλύτερος τρόπος να περνάει κανείς τα απογεύματά του, σίγουρα καλύτερο από τη χαρτοπαιξία και το ποτό. Ταυτόχρονα, μέ έναν ταπεινό εμπαιγμό προς τον εαυτό του, ξέρει πως αν μαθευτεί η ταυτότητά του, ο κακός κόσμος θα τον αποκαλεί \enquote{ο γάιδοαρος με δέρμα λιονταριού}. Ο χαρακτήρας του κυρίου αυτού είναι ένα πολύ ευτυχές μείγμα ήπιων και θυμώδων χαρακτηριστικών, κάτι που τον έκανε να ξεπερνάει κατά πολύ τους προκατόχους του και έχει καταφέρει να προσελκύσει τα μεγαλύτερα πλήθη στην ανθρώπινη ιστορία.

Δεν γίνεται να ολοκληρώσω το κείμενό μου, χωρίς να αναφερθώ στις αβάσιμες κατηγορίες που δέχεται το πρόσωπο ενός κυρίου, του οποίου του πρέπει να παραδεχτώ πως είμαι θαυμαστής· μιλάω για το γεγονός πως ο Signor Nicolini θεάθηκε μαζί με το λιοντάρι να καπνίζουν μαζί μία πίπα, στο παρασκήνιο. Με αυτό οι κοινοί εχθροί τους, θέλουν να υπονοήσουν πως η μάχη τους πάνω στη σκηνή είναι μία απάτη. Αλλά με την έρευνά μου, ανακάλυψα πως εάν υπήρξε τέτοια επαφή, έγινε αφού είχε ήδη προηγηθεί η μάχη, και όπως ορίζουν οι κανόνες τους δράματος το λιοντάρι πρέπει να θεωρείται νεκρό. Και τέλος πάντων, είναι ένα θέαμα που μπορεί ο καθένας να δει στο Westminster Hall, που δεν υπάρχει πιο συνηθισμένο θέαμα από δύο δικηγόρους που μετά το πέρας της δίκης, βρίσκονται αγκαλιασμένοι εκτός της αίθουσας.

Δεν θέλω όμως να σταθώ στον Signor Nicolini, που παίζοντας το μέρος του, απλά συμμορφώνεται με το απαίσιο γούστο του κοινού του. Ξέρει πολύ καλά, πως το λιοντάρι έχει πολύ περισσότερους θαυμαστές από τον ίδιο. Όπως λένε για το έφιππο άγαλμα στο Pont Neuf στο Παρίσι, περισσότερος κόσμος το επισκέπτεται για να δει το άλογο, παρά τον βασιλιά που το καβαλάει. Αντίθετα, μου προσφέρει δίκαια αγανάκτηση, να βλέπω έναν άνθρωπο που οι πράξεις του δίνουν εκ νέου μεγαλειότητα στους βασιλιάδες, αποφασιστικότητα στους ήρωες και τρυφερότητα στους εραστές, αλλά να βυθίζεται από το μεγαλείο της συμπεριφοράς του. Πολύ συχνά έυχομαι, οι σύγχρονοι τραγωδοί μας, να αντέγραφαν αυτόν τον σπουδαίο δάσκαλο. Να χρησιμοποιήσουν τα χέρια και τα πόδια τους, και να πάρουν τις κατάλληλες εκφράσεις με το πρόσωπό τους, με βλέμματα γεμάτα σημασία και πάθη. Πόσο λαμπρή θα ήταν η Αγγλική τραγωδία με τέτοια δράση, που είναι ικανή να προδώσει αξιοπρέπεια στις εξαναγκασμένες ιδέες, ψυχρές ματαιοδοξίες και αφύσικες αντιδράσεις της Ιταλικής όπερας. Ανέφερα τη μάχη με το λιοντάρι, ώστε να αναδείξω τη κυριαρχεί σήμερα στην εξευγενισμένη διασκέδαση της Μεγάλης Βρετανίας.

Το κοινό συχνά καταδικάζεται από τους συγγραφείς για τη χυδαιότητά του γούστου του, αλλά το παράπονο σήμερα φαίνεται να είναι, όχι η ανάγκη για καλό γούστο, παρά μόνο για κοινή λογική.