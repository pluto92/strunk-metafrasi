\translationsetup
    {Joseph Addison}
    {1711}
    {The Spectator}
    {Μανώλης Κυζάλας}

\pagestyle{texts}
\chapter[%
    \originalauthor\ -- \emph{\translatedtitle}\ \yearpublished\\%
    {\normalfont \emph{μετάφραση:} \maintranslator}%
]{\originalauthor}

\begin{preface}
    Ενάς από τους σπουδαιότερους συγγραφείς και ανθρώπους των γραμμάτων, ο Addison (1672-1719) συνείσφερε κείμενά του σε πολλά από τα περιοδικά της εποχής, όπως το \emph{The Tatler, The Spectator} και το \emph{The Guardian}, που δημοσιεύονταν από τον φίλο του Richard Steele, από το 1709 εώς το 1713. Ιδιαίτερα σημαντικές είναι οι συνεισφορές του Addison στο Spectator που κρατούσε μετριοπαθή στάση στους πολιτικούς διαξιφισμούς της εποχής. Στα κείμενα του ο Addison περιγράφει γλαφυρά τον τρόπο ζωής και συμπεριφοράς των σύγχρονών του. Τα έξυπνα κείμενά του είχαν σημαντική επιρροή στην χώρο της κριτικής, όχι μόνο σε Άγγλους συγγραφείς, αλλά και σε Γάλλους και Γερμανούς.
\end{preface}

\begin{center}
    \textbf{\translatedtitle}

    [αποσπάσματα]

    μετάφραση: \maintranslator
\end{center}

\begin{center}
    \emph{Τρίτη, 6 Μαρτίου, 1711.}

    \emph{Spectatum admissi risum teneatis?}%
    \footnote{Θα μπορούσατε, φίλοι μου, εάν είχατε την ευκαιρία για μία ιδιωτική θέαση, να μην γελάσετε; --- Οράτιος, \emph{Ars Poetica}}
\end{center}

Θα επιτρέπαμε σε μία όπερα να είναι υπερβολικά πληθωρική με τα στολίδια της, αφού ο μόνος στόχος της είναι να ευχαριστεί τις αισθήσεις, και να διατηρήσει την προσοχή ενός νωχελικού κοινού. Όπως πρόταση, όμως, η κοινή λογική, δεν θα πρέπει να υπάρχει τίποτα στη σκηνή και στις μηχανές που χρησιμοποιούνται που να φαίνεται παιδικό η άτοπο. Πόσο θα γελούσαν οι πνευματικοί άνθρωποι του καιρού του βασιλιά Καρόλου, εάν έβλεπαν τον Nicolini εκτεθειμένο σε μία καταιγίδα, ενώ ήταν ντυμένος με στολή ερμίνας, ενώ έπλεε σε μία θάλασσα από χαρτόνι! Πόσο θα τους χλεύαζαν εάν έβλεπαν ζωγραφισμένος δράκους να πετάνε φωτιές, μαγεμένα άρματα ζωγραφισμένα από φλαμανδικές φοράδες, και πραγματικούς καταρράκτες να χύνονται πάνω σε τεχνητά τοπία. Με ελάχιστη ικανότητα στην κριτική θα μπορούσαμε να καταλάβουμε πως οι σκιές και η πραγματικότητα δεν πρέπει να αναμειγνύονται μέσα στο ίδιο έργο· πως οι σκηνές που είναι σχεδιασμένες να παρουσιάζουν την φύση θα πρέπει να είναι γεμάτες αναπαραστάσεις και όχι με τα πραγματικά αντικείμενα αυτούσια. Εάν κάποιος ήθελε να αναπαραστήσει μία πλατιά πεδιάδα με αγέλες και κοπάδια, θα ήταν γελοίο να ζωγραφίσει την πεδιάδα πάνω στη σκηνή αλλά να πλημμυρίσει το σανίδι με πρόβατα και βόδια. Αν έκανε αυτό θα συνδύαζε ασύνδετα πράγματα και θα και θα έφτιαχνε ένα στήσιμο σκηνής εν μέρει αληθινό και εν μέρει φανταστικό. Θα πρότεινα αυτά που είπα παραπάνω σε όλους τους σκηνοθέτες, όπως και σε όλους τους θαυμαστές της σύγχρονης όπερας.

Όπως περπατούσα στους δρόμους πριν από ένα δεκαπενθήμερο, είδα έναν συνηθισμένο τύπο να κουβαλάει ένα κλουβί γεμάτο με μικρά πουλιά στους ώμους του· και καθώς αναρωτιόμουνα τι τα χρειάζεται, εντελώς τυχαία συναντήθηκε με ένα γνωστό του που είχε την ίδια απορία. Αφού τον ρώτησα τι κουβαλάει στους ώμους του, εκείνος του απάντησε πως αγοράζει σπουργίτια για την όπερα. Σπουργίτια για την όπερα! Είπε ο φίλος του, ενώ του τρέχανε τα σάλια, για να τα ψήσουνε; Όχι, όχι, λέει ο άλλος, Είναι για να Μπουν στο τέλος της πρώτης πράξεις και να πετάξουν εκτός σκηνής.

Αυτός ο παράξενος διάλογος μου κίνησε τόσο πολύ την περιέργεια που αγόρασα εισιτήρια για την όπερα, που πίστευα πως τα σπουργίτια θα τραγουδήσουν σε κάποιο ευχάριστο δάσος. Παρατηρώντας καλύτερα, βέβαια, κατάλαβα πως τα σπουργίτια προσπάθησα να με ξεγελάσουν όπως ξεγέλασε ο Sir Martin Mar-all την ερωμένη του, αφού ενώ αυτά πετούσαν μπροστά στα μάτια μου, η μουσική ερχόταν πίσω από τις κουρτίνες, με φλάουτα και καλέσματα πουλιών. Την ίδια στιγμή που έκανε αυτή την ανακάλυψη, έμαθα επίσης, από τις συζητήσεις των ηθοποιών, πως υπήρχαν μεγάλα σχέδια για την βελτίωση αυτής της όπερας. Είχε προταθεί να ρίξουν ένα μέρος του τοίχου και να ξαφνιάσουν το κοινό με εκατό άλογα και πως υπήρχε ένα εγχείρημα στα σκαριά, να φέρουν το New River μέσα στο κτίριο, ώστε να φτιάξουν πίδακες και άλλες κατασκευές με το νερό. Αυτό το εγχείρημα όπως έχω ακούσει, θα αναβληθεί μέχρι και το καλοκαίρι, όπου η ψύχρα και η υγρασία από τους πίδακες και τους καταρράχτες θα είναι πιο ευπρόσδεκτη από ανθρώπους επιπέδου. Εν το μεταξύ, για ένα πιο ταιριαστό θέαμα στον χειμώνα, η όπερα του Rinaldo είναι γεμάτη με βροντές, αστραπές και φωτιές. Το κοινό μπορεί να τα παρακολουθήσει χωρίς να κρυώσει, και σίγουρα δεν θα διατρέχει πολύ μεγάλο κίνδυνο να καεί· αφού υπάρχουν πολλές μηχανές γεμάτες νερό, και έτοιμες για λειτουργία σε περίπτωση τέτοιων ατυχημάτων. Παραταύτα, επειδή ο ιδιοκτήτης του κτιρίου είναι πολύ καλός μου φίλος, ελπίζω να ήταν αρκετά σοφός ώστε να το ασφαλίσει πριν παιχτεί αυτή η όπερα.

Σίγουρα αυτές οι σκηνές θα έχουν πάρα πολύ ενδιαφέρον, αφού τις έχουν γράψει δύο ποιητές διαφορετικής εθνικότητες και τις απαρτίζουν δύο μάγοι διαφορετικού φύλλου. Η Armida (όπως μαθαίνουμε από τον διάλογο) είναι αμαζόνα μάγισσα, ενώ ο φτωχός Signor Cassani (όπως μαθαίνουμε από τον χαρακτήρα του) είναι ένας χριστιανός μάγος (Mago Christiano). Πρέπει να ομολογήσω πως είμαι πολύ προβληματισμένος· πώς μία αμαζόνα μαθαίνει την τέχνη της μαύρης μαγίας, αλλά και πώς ένας καλός χριστιανός συνδιαλέγεται με τον διάβολο.

Ας αφουγκραστούμε και τους ποιητές μετά τους μάγους. Θα δώσω μία γεύση των Ιταλικών στις πρώτες γραμμές τις εισαγωγής. \en{Eccoti, benigno lettore, un parto di poche sere, che se ben nato di notte, non è pèro aborto di tenebre, mà si farà conoscere figliolo d'Apollo con qualche raggio di Parnasso}. \enquote{Δες, ευγενικέ αναγνώστη, τη γέννηση μερικών βραδιών, που, αν και είναι τα παιδιά της νύχτας, δεν είναι --- του σκοταδιού, αλλα θα φανερωθεί φανερωθεί ως ο γιος του Απόλονα με το φως του Παρνασού.} Στη συνέχεια αποκαλεί τον Mynheer Hendel \enquote{Ορφέα της εποχής μας}, και στη συνέχεια μας γνωστοποιεί, με την ίδια αιθεριότητα στο ύφος του, πως έγραψε την όπερα, μόλις σε ένα δεκαπενθήμερο. Αυτό είναι το πνέυμα αυτών, στων οποίων το γούστο τόσο φιλόδοξα επαναπαυόμαστε. Η αλήθεια Του ζητήματος είναι πως οι μεγαλύτεροι Ιταλοί συγγραφείς της εποχής μας εκφράζονται με ένα τόσο διανθισμένο λεξιλόγιο, και με μία τόσο ανιαρή περιφραστικότητα, που στην χώρα μας δεν συνηθίζει κανείς παρά μόνο οι πιο σχολαστικοί ακαδημαϊκοί, ενώ ταυτόχρονα γεμίζουν τα γραπτά τους με τόσο φτωχή φαντασία και ματαιοδοξία, που η νεολαία μας θα ντρεπόταν ακόμα και πριν τα πρώτα δύο χρόνια στο πανεπιστήμιο. Πολλοί θα βιαστούν να πουν πως είναι η διαφορά στη ευφυΐα που ευθύνεται για την διαφορά στο επίπεδο των έργων των δύο εθνών, μα για να αποδείξω πως δεν ισχύει κάτι τέτοιο, δεν χρειάζεται παρά να κοιτάξουμε στα γραπτά των παλιών Ιταλών, όπως του Cicero και του Βιργίλιου· Αυτό που θα βρούμε είναι πως οι άγγλοι συγγραφείς, στον τρόπο σκέψης τους και έκφρασής τους, είναι πολύ πιο κοντά στου παλιούς Ιταλούς, απ'ότι οι σύγχρονοι Ιταλοί που απλώς υποκρίνονται πως είναι. Και για τον ποιητή τον ίδιο, από του οποίου τα όνειρα δημιουργήθηκε αυτή η όπερα, θα συμφωνήσω απόλυτα με τον Monsieur Boileau, πως μία στροφή του Βιργίλιου, αξίζει όσο όλες οι χάντρες κι οι πούλιες του Tasso. 

Ας επιστρέψουμε όμως στα σπουργίτια. Υπήρξαν τόσες πτήσεις με τα αφηνιασμένα σπουργίτια σε αυτήν την όπερα, που είναι δύσκολο να βγάλουν πλέον όλα έξω. Μπαίνουν στα έργα που δεν έχουν καμία θέση, σε πολύ ακατάλληλες σκηνές· βρίσκονται σε κρεβατοκάμαρες κυριών, ή κάθονται στον θρόνο του βασιλία· και πόσο υποφέρουν τα κεφάλια του κοινού. Έμπιστες πηγές μου έχουν μεταφέρει πως μία φορά έγινε η προσπάθεια να ανέβει η ιστορία του Whittington και της γάτας του, και για την προσπάθεια αυτή, μάζεψαν πάρα πολλά ποντίκια. Ο κύριος Rich όμως, ο ιδιοκτήτης του κτιρίου, πολύ σοφά σκέφτηκε πως θα ήταν αδύνατο για τη γάτα να τα σκοτώσει όλα, και συνεπώς οι πρίγκιπες στην σκηνή θα είχαν τόσα ποντίκια όσο ο πρίγκιπας στο νησί, πριν την άφιξη της γάτας, και για αυτό δεν το επέτρεψε τελικά. Και φυσικά δεν μπορώ να τον κατηγορήσω, αφού όπως είπε, δεν ήξερε κάποιον από τους μουσικούς στην όπερα, τόσο ικανό όσο τον γνωστό pied piper, που έκανε όλα τα ποντικιά μίας μεγάλης πόλης στην γερμανία, να ακολουθήσουν τη μουσική του, και έτσι καθάρισε όλη την περιοχή από περίεργα μικρά ζώα.

Before I dismiss this paper, I must inform my reader, that I hear there is a treaty on foot with London and Wise (who will be appointed gardeners of the playhouse) to furnish the opera of Rinaldo and Armida with an orange-grove; and that the next time it is acted, the singing birds will be personated by tom-tits: the undertakers being resolved to spare neither pains nor money for the gratification of the audience.

\begin{center}
    \emph{Thursday, March 15, 1711.}

    \emph{Die mihi, nfills tu leo, qualis eris}
\end{center}

There is nothing that of late years has afforded matter of greater amuse- ment to the town than Signor Nicolini's combat with a lion 11 in the Hay- market, which has been very often exhibited to the general satisfaction of most of the nobility and gentry in the kingdom of Great Britain. Upon the first rumor of this intended combat, it was confidently affirmed, and is still believed by many in both galleries, that there would be a tame lion sent from the Tower every opera night, in order to be killed by Hydaspes; this report, though altogether groundless, so universally pre- vailed in the upper regions of the playhouse, that some of the most refined politicians in those parts of the audience gave it out in whisper, that the lion was a cousin-german of the tiger who made his appearance in King William's days, and that the stage would be supplied with lions at the public expense, during the whole session. Many likewise were the con- jectures of the treatment which this lion was to meet with from the hands of Signor Nicolini: some supposed that he was to subdue him in recitativo, as Orpheus used to serve the wild beasts in his time, and afterwards to knock him on the head; some fancied that the lion would not pretend to lay his paws upon the hero, by reason of the received opinion, that a lion will not hurt a virgin; several, who pretended to have seen the opera in Italy, had informed their friends, that the lion was to act a part in High- Dutch, and roar twice or thrice to a thorough bass, before he fell at the feet of Hydaspes. To clear up a matter that was so variously reported, I have made it my business to examine whether this pretended lion is really the savage he appears to be, or only a counterfeit.


But before I communicate my discoveries, I must acquaint the reader, that upon my walking behind the scenes last winter, as I was thinking on something else, I accidentally justled against a monstrous animal that extremely startled me, and upon my nearer survey of it, appeared to be a lion rampant. The lion, seeing me very much surprised, told me, in a gentle voice, that 1 might come by him if I pleased: "For," says he, "I do not intend to hurt anybody." I thanked him very kindly, and passed by him. And in a little time after saw him leap upon the stage, and act his part with very great applause. It has been observed by several, that the lion has changed his manner of acting twice or thrice since his first appear- ance; which will not seem strange, when I acquaint my reader that the lion has been changed upon the audience three several times. The first lion was a candle-snuffer, who being a fellow of a testy, choleric temper, overdid his part, and would not suffer himself to be killed so easily as he ought to have done; besides, it was observed of him, that he grew more surly every time he came out of the lion, and having dropped some words in ordinary conversation, as if he had not fought his best, and that he suffered himself to be thrown upon his back in the scufHe, and that he would wrestle with Mr. Nicolini for what he pleased, out of his lion's skin, it was thought proper to discard him; and it is verily believed, to this day, that had he been brought upon the stage another time, he would certainly have done mischief. Besides, it was objected against the first lion, that he reared himself so high upon his hinder paws, and walked in so erect a posture, that he looked more like an old man than a lion.

The second lion was a tailor by trade, who belonged to the playhouse nd had the character of a mild and peaceable man in his profession. If the former was too furious, this was too sheepish for his part; insomuch that after a short, modest walk upon the stage, he would fall at the first touch of Hydaspes, without grappling with him, and giving him an opportunity of showing his variety of Italian trips. It is said, indeed, that he once gave him a rip in his flesh-colored doublet; but this was only to make work for himself, in his private character of a tailor. I must not omit that it was this second lion who treated me with so much humanity behind the scenes.

The acting lion at present is, as I am informed, a country gentleman, who does it for his diversion, but desires his name may be concealed. He says, very handsomely, in his own excuse, that he does not act for gain; that he indulges an innocent pleasure in it; and that it is better to pass away an evening in this manner than in gaming and drinking: but at the same time says, with a very agreeable raillery upon himself, that if his name should be known, the ill-natured world might call him, "the ass in the lion's skin." This gentleman's temper is made out of such a happy mixture of the mild and choleric, that he outdoes both his predecessors, and has drawn together greater audiences than have been known in the memory of man.

I must not conclude my narrative, without taking notice of a groundless report that has been raised to a gentleman's disadvantage, of whom I must declare myself an admirer; namely, that Signor Nicolini and the lion have been seen sitting peaceably by one another, and smoking a pipe together behind the scenes; by which their common enemies would insinuate, that it is but a sham combat which they represent upon the stage: but upon inquiry I find, th~t if any such correspondence has passed between them, it was not till the combat was over, when the lion was to be looked upon as dead, according to the received rules of the drama. Besides, this is what is practised every day in Westminster Hall, where nothing is more usual than to see a couple of lawyers, who have been tearing each other to pieces in the court, embracing one another as soon as they are out of it.

I would not be thought, in any part of this relation, to reflect upon Signor Nicolini, who in acting this part only complies with the wretched taste of his audience; he knows very well, that the lion has many more admirers than himself; as they say of the famous equestrian statue on the Pont Neuf at Paris, that more people go to 'see the horse than the king who sits upon it. On the contrary, it gives me a just indignation to see a person whose action gives new majesty to kings, resolution to heroes, and softness to lovers, thus sinking from the greatness of his behavior, and degraded into the character of the London Prentice. I have often wished, that our tragedians would copy after this great master in action. Could they make the same use of their arms and legs, and inform their faces with as significant looks and passions, how glorious would an English tragedy appear with that action which is capable of giving a dignity to the forced thoughts, cold conceits, and unnatural expressions of an Italian opera! In the meantime, I have related this combat of the lion, to show what are at present the reigning entertainments of the politer part of Great Britain.

Audiences have often been reproached by writers for the coarseness of their taste; but our present grievance does not seem to be the want of a good taste, but of common sense.