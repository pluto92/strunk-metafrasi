\translationsetup
    {Joseph Addison}
    {1711}
    {The Spectator}
    {Μανώλης Κυζάλας}

\pagestyle{texts}
\chapter
    [\originalauthor\ -- \emph{\translatedtitle} (αποσπάσματα)  -- \yearpublished]
    {\originalauthor}

\begin{preface}
    Ενάς από τους σπουδαιότερους συγγραφείς και ανθρώπους των γραμμάτων, ο Addison (1672-1719) συνείσφερε κείμενά του σε πολλά από τα περιοδικά της εποχής, όπως το \emph{The Tatler, The Spectator} και το \emph{The Guardian}, που δημοσιεύονταν από τον φίλο του Richard Steele, από το 1709 εώς το 1713. Ιδιαίτερα σημαντικές είναι οι συνεισφορές του Addison στο Spectator που κρατούσε μετριοπαθή στάση στους πολιτικούς διαξιφισμούς της εποχής. Στα κείμενα του ο Addison περιγράφει γλαφυρά τον τρόπο ζωής και συμπεριφοράς των σύγχρονών του. Τα έξυπνα κείμενά του είχαν σημαντική επιρροή στην χώρο της κριτικής, όχι μόνο σε Άγγλους συγγραφείς, αλλά και σε Γάλλους και Γερμανούς.
\end{preface}

\begin{center}
    \textbf{\translatedtitle}

    [αποσπάσματα]

    μετάφραση: \maintranslator
\end{center}

\begin{center}
    \emph{Tuesday, March 6, 1711.}

    \emph{Spectatum admissi risum teneatis?}
\end{center}

AN OPERA may be allowed to be extravagantly lavish in its decorations, as its only design is to gratify the senses, and keep up an indolent attention in the audience. Common sense, however) requires, that there should be nothing in the scenes and machines which may appear childish and absurd. How would the wits of King Charles's time have laughed to have seen Nicolini 8 exposed to a tempest in robes of ermine, and sailing in an open boat upon a sea of pasteboard! What a field of raillery would they have been let into, had they been entertained with painted dragons spitting wildfire, enchanted chariots drawn by Flanders mares, and real cascades in artificial landscapes! 4 A little skill in criticism would inform us, that shadows and realities ought not to be mixed together in the same piece; and that the scenes which are designed as the representations of nature, should be filled with resemblances, and not with the things themselves. If one would represent a wide champaign country filled with herds and Bocks, it would be ridiculous to draw the country only upon the scenes, and to crowd several parts of the stage with sheep and oxen. This is joining together inconsistencies, and making the decoration partly real and partly imaginary. I would recommend what I have here said to the directors, as well as to the admirers, of our modern opera.

As I was walking in the streets about a fortnight ago, I saw an ordinary fellow carrying a cage full of little birds upon his shoulder; and, as I was wondering with myself what use he would put them to, he was met very luckily by an acquaintance, who had the same curiosity. Upon his asking him what he had upon his shoulder, he told him, that he had been buying sparrows for the opera. Sparrows for the opera! says his friend, licking his lips; what, are they to be roasted? No, no, says the other; they are to enter towards the end of the first act, and to fly about the stage.

This strange dialogue awakened my curiosity so far, that I immediately bought the opera, by which means I perceived the sparrows were to act the part of singing birds in a delightful grove; though, upon a nearer inquiry, I found the sparrows put the same trick upon the audience, that Sir Martin Mar-all G practised upon his mistress; for, though they flew in sight, the music proceeded from a consort of flagelets and bird-calls 6 which were planted behind the scenes. At the same time I made this dis- covery, I found, by the discourse of the actors, that there were great de- signs on foot for the improvement of the opera; that it had been proposed to break down a pa~tof the wall, and to surprise the audience with a party of an hundred horse; and that there was actually a project of bringing the New River into the house, to be employed in jetteaus and water-works. This project, as I have since heard, is postponed till the summer season; when it is thought the coolness that proceeds from fountains and cascades will be more acceptable and refreshing to people of quality. In the mean- time, to find out a more agreeable entertainment for the winter season, the opera of Rinaldo is filled with thunder and lightning, illuminations and fire-works; which the audience may look upon without catching cold, and indeed without much danger of being burnt; for there are several engines filled with water, and ready to play at a minute's warning, in case any such accident should happen. However, as I have a very great friend- ship for the owner of this theatre, I hope that he has been wise enough to insure his house before he would let this opera be acted in it.

It is no wonder that those scenes should be very surprising, which were contrived by two poets of different nations,T and raised by two magicians of different sexes. Armida (as we are told in the argument) was an Amazonian enchantress, and poor Signor Cassani (as we learn from the persons represented) a Christian conjuror (Mago Christiano). I must confess I am very much puzzled to find how an Amazon should be versed in the black art; or how a good Christian (for such is the part of the magician) should deal with the devil.

To consider the poets after the conjurors, I shall give you a taste of the Italian from the first lines of his preface: Eccoti, benigno lettore, un parto di poche sere, che se ben nato di notte, non eperC> aborto di tenebre, rna si fara conoscere figliolo d'Apollo con qualche raggio di Parnasso. "Behold, gentle reader, the birth of a few evenings, which, though it be the offspring of the night, is not the abortive of darkness, but will make itself known to be the son of Apollo, with a certain ray of Parnassus." He afterwards proceeds to call Mynheer Hendel the Orpheus of our age, and to acquaint us, in the same sublimity of style, that he composed this opera in a fortnight. Such are the wits to whose tastes we so ambitiously conform ourselves. The truth of it is, the finest writers among the modern Italians express themselves in such a florid form of words, and such tedious circumlocutions, as are used by none but pedants in our own country; and at the same time fill their writings with such poor imagina- tions and conceits, as our youths are ashamed of before they have been two years at the university. Some may be apt to think that it is the difference of genius which produces this difference in the works of the two nations; but to show there is nothing in this, if we look into the writings of the old Italians, such as Cicero and Virgil, we shall find that the English writers, in their way of thinking and expressing themselves, resemble those authors much more than the modern Italians pretend to do. And as for the poet himself,8 from whom the dreams of this opera are taken, I must entirely agree with Monsieur Boileau, that one verse in Virgil is worth all the clinquant or tinsel of Tasso.'

But to return to the sparrows; there have been so many flights of them let loose in this opera, that it is feared the house will never get rid of them; and that in other plays they make their entrance in very wrong and improper scenes, so as to be seen flying in a lady's bed-chamber, or perching upon a king's throne; besides the inconveniences which the heads of the audience may sometimes suffer from them. I am credibly informed, that there was once a design of casting into an opera the story of Whitting- ton and his cat, and that in order to do it, there had been got together a great quantity of mice; but Mr. Rich, the proprietor of the playhouse, very prudently considered, that it would be impossible for the cat to kill them all, and that consequently the princes of the stage might be as much infested with mice, as the prince of the island was before the eat's arrival upon it; for which reason he would not permit it to be acted in his house. And indeed I cannot blame him: for, as he said very well upon that occasion, I do not hear that any of the performers in our opera pretend to equal the famous pied piper, who made all the mice of a great town in Germany follow his music, and by that means cleared the place of those little noxious animals.

Before I dismiss this paper, I must inform my reader, that I hear there is a treaty on foot with London and Wise (who will be appointed gardeners of the playhouse) to furnish the opera of Rinaldo and Armida with an orange-grove; and that the next time it is acted, the singing birds will be personated by tom-tits: the undertakers being resolved to spare neither pains nor money for the gratification of the audience.

\begin{center}
    \emph{Thursday, March 15, 1711.}

    \emph{Die mihi, nfills tu leo, qualis eris}
\end{center}

There is nothing that of late years has afforded matter of greater amuse- ment to the town than Signor Nicolini's combat with a lion 11 in the Hay- market, which has been very often exhibited to the general satisfaction of most of the nobility and gentry in the kingdom of Great Britain. Upon the first rumor of this intended combat, it was confidently affirmed, and is still believed by many in both galleries, that there would be a tame lion sent from the Tower every opera night, in order to be killed by Hydaspes; this report, though altogether groundless, so universally pre- vailed in the upper regions of the playhouse, that some of the most refined politicians in those parts of the audience gave it out in whisper, that the lion was a cousin-german of the tiger who made his appearance in King William's days, and that the stage would be supplied with lions at the public expense, during the whole session. Many likewise were the con- jectures of the treatment which this lion was to meet with from the hands of Signor Nicolini: some supposed that he was to subdue him in recitativo, as Orpheus used to serve the wild beasts in his time, and afterwards to knock him on the head; some fancied that the lion would not pretend to lay his paws upon the hero, by reason of the received opinion, that a lion will not hurt a virgin; several, who pretended to have seen the opera in Italy, had informed their friends, that the lion was to act a part in High- Dutch, and roar twice or thrice to a thorough bass, before he fell at the feet of Hydaspes. To clear up a matter that was so variously reported, I have made it my business to examine whether this pretended lion is really the savage he appears to be, or only a counterfeit.


But before I communicate my discoveries, I must acquaint the reader, that upon my walking behind the scenes last winter, as I was thinking on something else, I accidentally justled against a monstrous animal that extremely startled me, and upon my nearer survey of it, appeared to be a lion rampant. The lion, seeing me very much surprised, told me, in a gentle voice, that 1 might come by him if I pleased: "For," says he, "I do not intend to hurt anybody." I thanked him very kindly, and passed by him. And in a little time after saw him leap upon the stage, and act his part with very great applause. It has been observed by several, that the lion has changed his manner of acting twice or thrice since his first appear- ance; which will not seem strange, when I acquaint my reader that the lion has been changed upon the audience three several times. The first lion was a candle-snuffer, who being a fellow of a testy, choleric temper, overdid his part, and would not suffer himself to be killed so easily as he ought to have done; besides, it was observed of him, that he grew more surly every time he came out of the lion, and having dropped some words in ordinary conversation, as if he had not fought his best, and that he suffered himself to be thrown upon his back in the scufHe, and that he would wrestle with Mr. Nicolini for what he pleased, out of his lion's skin, it was thought proper to discard him; and it is verily believed, to this day, that had he been brought upon the stage another time, he would certainly have done mischief. Besides, it was objected against the first lion, that he reared himself so high upon his hinder paws, and walked in so erect a posture, that he looked more like an old man than a lion.

The second lion was a tailor by trade, who belonged to the playhouse nd had the character of a mild and peaceable man in his profession. If the former was too furious, this was too sheepish for his part; insomuch that after a short, modest walk upon the stage, he would fall at the first touch of Hydaspes, without grappling with him, and giving him an opportunity of showing his variety of Italian trips. It is said, indeed, that he once gave him a rip in his flesh-colored doublet; but this was only to make work for himself, in his private character of a tailor. I must not omit that it was this second lion who treated me with so much humanity behind the scenes.

The acting lion at present is, as I am informed, a country gentleman, who does it for his diversion, but desires his name may be concealed. He says, very handsomely, in his own excuse, that he does not act for gain; that he indulges an innocent pleasure in it; and that it is better to pass away an evening in this manner than in gaming and drinking: but at the same time says, with a very agreeable raillery upon himself, that if his name should be known, the ill-natured world might call him, "the ass in the lion's skin." This gentleman's temper is made out of such a happy mixture of the mild and choleric, that he outdoes both his predecessors, and has drawn together greater audiences than have been known in the memory of man.

I must not conclude my narrative, without taking notice of a groundless report that has been raised to a gentleman's disadvantage, of whom I must declare myself an admirer; namely, that Signor Nicolini and the lion have been seen sitting peaceably by one another, and smoking a pipe together behind the scenes; by which their common enemies would insinuate, that it is but a sham combat which they represent upon the stage: but upon inquiry I find, th~t if any such correspondence has passed between them, it was not till the combat was over, when the lion was to be looked upon as dead, according to the received rules of the drama. Besides, this is what is practised every day in Westminster Hall, where nothing is more usual than to see a couple of lawyers, who have been tearing each other to pieces in the court, embracing one another as soon as they are out of it.

I would not be thought, in any part of this relation, to reflect upon Signor Nicolini, who in acting this part only complies with the wretched taste of his audience; he knows very well, that the lion has many more admirers than himself; as they say of the famous equestrian statue on the Pont Neuf at Paris, that more people go to 'see the horse than the king who sits upon it. On the contrary, it gives me a just indignation to see a person whose action gives new majesty to kings, resolution to heroes, and softness to lovers, thus sinking from the greatness of his behavior, and degraded into the character of the London Prentice. I have often wished, that our tragedians would copy after this great master in action. Could they make the same use of their arms and legs, and inform their faces with as significant looks and passions, how glorious would an English tragedy appear with that action which is capable of giving a dignity to the forced thoughts, cold conceits, and unnatural expressions of an Italian opera! In the meantime, I have related this combat of the lion, to show what are at present the reigning entertainments of the politer part of Great Britain.

Audiences have often been reproached by writers for the coarseness of their taste; but our present grievance does not seem to be the want of a good taste, but of common sense.